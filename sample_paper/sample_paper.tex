\documentclass{amsart}

\usepackage[margin=1in]{geometry}
\usepackage{amsthm,amssymb, amsfonts}
\usepackage[utf8]{inputenc} % convert unicode to ascii
\usepackage{mathtools} % extension of AMS math
\usepackage{mathrsfs} % script letters
\usepackage{enumitem} % more control over enumerate environment
\usepackage{xcolor} % color text
\usepackage{graphicx} % include graphics
\usepackage{float} % better control over figures
\usepackage{fullpage} % remove excess whitespace
\usepackage[linktocpage]{hyperref} % hyperlink
\usepackage{tensor} % for tensor indices
\usepackage{wasysym} % for Box symbol (wave operator)
\usepackage{algpseudocode} % for pseudocode
\usepackage{epstopdf} % convert EPS files


% theorem style
\theoremstyle{plain}
\newtheorem{Th}{Theorem}[section]
\newtheorem{Lemma}[Th]{Lemma}
\newtheorem{Cor}[Th]{Corollary}
\newtheorem{prop}[Th]{Proposition}

% I don't like definitions nor remarks nor examples in italics, so I'll change those (+add in something to break up the spacing once the remark/example are done).
\theoremstyle{definition}
\newtheorem{definition}[Th]{Definition} 
\newtheorem{remarkx}[Th]{Remark}
\newenvironment{remark}
  {\pushQED{\qed}\renewcommand{\qedsymbol}{$\triangle$}\remarkx}
  {\popQED\endremarkx}
\newtheorem{examplex}[Th]{Example}
\newenvironment{example}
  {\pushQED{\qed}\renewcommand{\qedsymbol}{$\triangle$}\examplex}
  {\popQED\endexamplex}  


% plain text math symbols

\newcommand{\imag}{\operatorname{Im}}
\newcommand{\real}{\operatorname{Re}}
\newcommand{\sgn}{\operatorname{sgn}}
\newcommand{\supp}{\operatorname{supp}}
\newcommand{\wf}{\operatorname{WF}}
\newcommand{\ran}{\operatorname{ran}}
\newcommand{\coker}{\operatorname{coker}}
\newcommand{\trace}{\operatorname{Tr}}

% blackboard bold for standard sets

\newcommand{\R}{\mathbb{R}}
\newcommand{\Q}{\mathbb{Q}}
\newcommand{\C}{\mathbb{C}}
\newcommand{\F}{\mathbb{F}}
\newcommand{\Z}{\mathbb{Z}}
\newcommand{\N}{\mathbb{N}}
\newcommand{\T}{\mathbb{T}}

% calligraphic for sets, spaces, etc.

\newcommand{\calA}{\mathcal{A}}
\newcommand{\calB}{\mathcal{B}}
\newcommand{\calC}{\mathcal{C}}
\newcommand{\calD}{\mathcal{D}}
\newcommand{\calE}{\mathcal{E}}
\newcommand{\calF}{\mathcal{F}}
\newcommand{\calK}{\mathcal{K}}
\newcommand{\calL}{\mathcal{L}}
\newcommand{\calM}{\mathcal{M}}
\newcommand{\calN}{\mathcal{N}}
\newcommand{\bigO}{\mathcal{O}}
\newcommand{\calS}{\mathcal{S}}
\newcommand{\calT}{\mathcal{T}}

% script for sets, spaces, etc.

\newcommand{\scrB}{\mathscr{B}}
\newcommand{\scrC}{\mathscr{C}}
\newcommand{\scrD}{\mathscr{D}}
\newcommand{\scrS}{\mathscr{S}}
\newcommand{\scrT}{\mathscr{T}}

% bold for vectors

\newcommand{\boldu}{\mathbf{u}}
\newcommand{\boldv}{\mathbf{v}}
\newcommand{\boldw}{\mathbf{w}}
\newcommand{\boldx}{\mathbf{x}}
\newcommand{\boldy}{\mathbf{y}}
\newcommand{\boldz}{\mathbf{z}}
\newcommand{\boldp}{\mathbf{p}}
\newcommand{\boldq}{\mathbf{q}}
\newcommand{\bolde}{\mathbf{e}}
\newcommand{\boldk}{\mathbf{k}}

\newcommand{\norm}[1]{\left\lVert #1 \right\rVert} % nice norm formatting (with size adjustments)
\newcommand{\normsub}[2]{\left\lVert #1 \right\rVert_{#2}} % nice norm formatting (with size adjustments) with subscript
\newcommand{\jbrac}[1]{\left\langle #1 \right\rangle} % nice Japanese bracket formatting (with size adjustments)
\newcommand{\inprod}[2]{\left\langle #1, #2 \right\rangle} % nice inner product  formatting (with size adjustments)
\newcommand{\posbrac}[2]{\left\lbrace #1, #2 \right\rbrace} % nice Poisson bracket formatting (with size adjustments)

\definecolor{mypink1}{rgb}{0.858, 0.188, 0.478} % actual pink color

\numberwithin{equation}{section}
\numberwithin{figure}{section}
\pagestyle{plain}
\title{Sample \LaTeX\ Paper/Project Write-up Format}
\author{{{Collin Kofroth}}}
\date{1/10/2022}

\begin{document}

\maketitle

\begin{abstract}

Here's a sample \LaTeX\ template. I'll highlight a few things that I do differently.

\end{abstract}

\tableofcontents

%$$P=(D_\alpha +A_\alpha) g^{\alpha\beta} (D_\beta+A_\beta)+iaD_t+V(t,x),$$ where $a, A_\alpha$ are real-valued, and $g,a, A$ are functions of $x$ alone. We might as well just consider $$P=D_\alpha g^{\alpha\beta}D_\beta+iaD_t.$$ If we assume that $\partial_t$ is uniformly time-like, we may divide through by $-g^{00}$ and preserve our assumptions on the operator coefficients. Hence, WLOG, we may assume that $g^{00}=-1$. The principal symbol of this operator is $$p(\tau, x,\xi)=\tau^2-2g^{j0}\tau\xi_j-g^{ij}\xi_i\xi_j.$$ 

\section{A Section}

This is how you generate a section. A few things that I want to point out are how the theorem environments look and differ (some are not in italics), and how I modified the remark environment (as well as the example one, which is not shown but similar).

\subsection{A Subsection}

This is how you generate a subsection. Here's a theorem.

\begin{Th}[Smoothness of Eigenfunctions, Lemma 14.5 in \cite{Zw}] \label{efun smooth}
Let $z\in\C$ and $u\in L^2(M)$ be such that $$(-h^2\Delta_g-z)u=0,$$ interpreted in the sense of distributions. Then, $u\in C^\infty(M)$.
\end{Th}

Here's a corollary.

\begin{Cor}
If $z$ is an eigenvalue, then the corresponding eigenfunction is smooth. 
\end{Cor}

\begin{proof}[Proof of Theorem \ref{efun smooth}]
I don't want to prove this right now.
\end{proof}

Moving on!

\subsection{Another Subsection}

The definitions look a bit different.

\begin{definition}[Theorem 3.6 in \cite{Zw}]\label{ana_ext}
Let $f\in \calS(\R).$  We say that $\tilde{f}$ is an analytic extension of $f$ to the complex plane provided that
\begin{enumerate}
\item $\tilde{f}\in C^\infty(\C),$ $\tilde{f}\big|_{\R}=f$ \item $\supp\tilde{f}\subset \{z\in\C: |\imag z|\leq 1\}$
\item $\bar{\partial}_z\tilde{f}(z)=\bigO(|\imag z|^\infty),$ where $\displaystyle{\bar{\partial}_z=\frac{1}{2}(\partial_x+i\partial_y)}$ is the Cauchy-Riemann operator.
\end{enumerate}
\end{definition}

I did not want the definitions to be italicized. Here is one more relevant type of mathematical statement.

\begin{prop}
Let $f\in\calS(\R^n),$ and fix a cutoff $\chi\in C_c^\infty((-1,1))$ such that $\chi\equiv 1$ on $[-1/2,1/2].$  Then, $$\tilde{f}(z):=\frac{1}{2\pi}\chi(y)\int\limits_{\R}\chi(y\xi)\hat{f}(\xi)e^{i\xi(x+iy)}\, d\xi$$ is an almost-analytic extension of $f$ to the complex plane (in the sense of Definition \ref{ana_ext}).  
\end{prop}

\section{Another Section}

Remarks also look a bit different.

\begin{remark}
Here is a remark on something, probably a very insightful comment. This puts a triangle at the end to indicate the end of the remark, which can be useful since remarks can be long and otherwise blend into the text.
\end{remark}

One last thing is the bibliography. I'm using bibtex; see the file ``biblio.bib." One must explicitly cite the articles for them to be shown. I will show some different types here (\cite{Lebeau}, \cite{Mor1,Mor2,Mor3}, \cite{MorWun},\cite{Perry},\cite{Tat1}), but I will omit one item from the aforementioned file (to demonstrate that it will not show up)

\section*{Acknowledgments}
I want to thank my poor handwriting and past graders for making me learn LaTeX earlier than many others.

\appendix

\section{Pseudocode}
Here's how you can do pseudocode:

\vspace*{.2in}

\begin{center}
\begin{algorithmic}
\State Solve $Ax_0= c$
\State $r\gets Bx_0-c$
\State tol$\gets 1e$-6
\State $i\gets 0$
\While{$\norm{r_i}>tol$}
\State Solve $Ad_i=r_i$
\State $x_{i+1}\gets x_i-d_i$
\State $i$++
\EndWhile
\end{algorithmic}
\end{center}

\bibliographystyle{amsplain}
\bibliography{biblio}
\end{document}

